% Definición del documento
\documentclass{article}
% Paquete para colores en tablas
\usepackage{colortbl}
% Paquete para las imágenes
\usepackage{graphicx}
% Paquete para la codificación UTF-8
\usepackage[utf8]{inputenc}
% Paquete para márgenes del documento
\usepackage[
  a4paper,
  left=2cm,
  right=2cm,
  top=2cm,
  bottom=3cm
]{geometry}
% Paquete para colores estándar y personalizados
\usepackage[dvipsnames]{xcolor}
\definecolor{fuchsia700}{rgb}{0.635, 0.109, 0.686}

%
%
%
%
%
% variables comunes y renombrado de comandos comunes
\newcommand{\dateEs}{Marzo 14, 2025} % variable fecha
\renewcommand{\arraystretch}{1.5}  % Espaciado vertical de filas

%
%
%
%
% 
\title{
  \textbf{
    \Huge
    \textcolor{darkgray}{Base de Datos}
    \\ \vspace{1cm}
    \huge
    \textcolor{darkgray}{Oracle Database Express Edition (XE)}
  }
}
\author{Alex Meza}
\date{\dateEs}
\begin{document}
\maketitle

%
%
%
%
%
% tabla de contenidos
\newpage
\tableofcontents

%
%
%
%
%
% historial de versiones
\newpage
\section{Historial de Versiones}
\vspace{0.5cm}
\begin{tabular}{| p{2cm} | p{3.4cm} | p{4.8cm} | p{5.6cm} |}
  \hline
    \rowcolor{fuchsia700}
    \textcolor{White}{\textbf{Versión}} &
    \textcolor{White}{\textbf{Fecha}} &
    \textcolor{White}{\textbf{Elaboró}} &
    \textcolor{White}{\textbf{Comentarios}}
  \\
  \hline
    0.0.1 &
    \today &
    Alex Meza \newline alexdanielmeza@gmail.com &
    Primera versiones
  \\
  \hline
\end{tabular}

%
%
%
%
%
% Objetivos
\newpage
\section{Objetivos}
\subsection{Objetivos Generales}
\subsection{Objetivos Específicos}

%
%
%
%
%
\section{Oracle Database Express Edition (XE)}
\subsection{Versiones de Oracle XE que aplican para esta documentación}
\subsection{Capacidades máximas de Oracle XE}

%
%
%
%
%
% Fin del documento
\end{document}